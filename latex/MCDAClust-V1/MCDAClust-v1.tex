%\documentclass{iosart2c}
\documentclass[]{elsarticle}

\usepackage{natbib}
\usepackage{amsmath,amsthm,mathtools}
\usepackage{amssymb}
\usepackage{mathtools, cuted}

\usepackage[table, dvipsnames]{xcolor}
\usepackage{verbatim}

\usepackage{mathrsfs}
\usepackage{stfloats}
\usepackage{tabularx}
\usepackage{subfigure}

\usepackage{xtab,booktabs}
\usepackage{arydshln}
\usepackage{multirow}
\usepackage{tikz,ifthen,calc}
\usepackage{pgfplots}
\pgfplotsset{compat=1.11}
\usepackage{tkz-fct}
\usepackage{color,colortbl}
\usepackage[labelfont=bf]{caption}

\usepackage{collcell}
\usepackage{hhline}
\usepackage{pgf}

\def\colorModel{hsb} %You can use rgb or hsb

\newcommand\ColCell[1]{
  \pgfmathparse{#1<1?1:0}  %Threshold for changing the font color into the cells
    \ifnum\pgfmathresult=0\relax\color{white}\fi
  \pgfmathsetmacro\compA{0}      %Component R or H
  \pgfmathsetmacro\compB{#1} %Component G or S
  \pgfmathsetmacro\compC{1}      %Component B or B
  \edef\x{\noexpand\centering\noexpand\cellcolor[\colorModel]{\compA,\compB,\compC}}\x #1
  } 
\newcolumntype{E}{>{\collectcell\ColCell}m{0.4cm}<{\endcollectcell}}  %Cell width


\usetikzlibrary{graphdrawing}
%\usetikzlibrary{graphicx}
%\usegdlibrary{trees}
\usetikzlibrary{shapes,decorations,arrows}
\usetikzlibrary{positioning}
\usetikzlibrary{arrows.meta}


\usepackage{enumitem}
\usepackage{algorithmic}
\usepackage{graphicx}
%\usepackage{subfig,graphicx}
\usepackage[titletoc,title]{appendix}

%\usepackage[blocks]{authblk}
%\newenvironment{keywords}{\noindent\textbf{Keywords:}}{}

\theoremstyle{definition}
\newtheorem*{inner}{\innerheader}
\newcommand{\innerheader}{}
\newenvironment{defi}[1]
 {\renewcommand\innerheader{#1}\begin{inner}}
 {\end{inner}}

\tolerance=1
\emergencystretch=\maxdimen
\hyphenpenalty=10000
\hbadness=10000

\newtheorem{prop}{Theorem}

\newtheorem{cond}{Condition}

\newcolumntype{d}[1]{D{.}{.}{#1}}



\journal{}
\bibliographystyle{model2-names}\biboptions{authoryear}

\begin{document}
\begin{frontmatter}                           % The preamble begins here.


\title{Stochastic clustering in the Human Development Index problem based on a non-compensatory and $K$-means approach}

%\runningtitle{Hesitant Fuzzy Sets and ELECTRE TRI-C}
%\subtitle{Subtitle}


%\author[A]{\fnms{Javier} \snm{Pereira}\thanks{Corresponding author. E-mail: jpereirar@inacap.cl.}},
%\author[B]{\fnms{Elaine C.B.} \snm{de Oliveira}}
%\author[C]{\fnms{Danielle C.} \snm{Morais}}
%\author[C]{\fnms{Ana Paula C.S.} \snm{Costa}}
%and
%\author[C]{\fnms{Luciana H.} \snm{Alencar}}
%
%\runningauthor{F. Author et al.}
%\address[A]{Universidad Tecnol\'ogica de Chile INACAP,  Santiago, Chile\\
%E-mail: xjavierpereira7@gmail.com}
%\address[B]{Instituto Federal da Paraiba, Campus Jo\~ao Pessoa, Jo\~ao Pessoa, Brazil\\
%E-mail: elainecjz@gmail.com}
%\address[C]{Universidade Federal de Pernambuco, Recife, Brazil\\
%E-mail: \{dcmorais,apcabral,lhalencar\}@cdsid.org.br}

\author[utc]{Javier Pereira$^{*,}$\cortext[cor1]}
\author[ap]{Pedro Contreras}
\author[ufpe]{Danielle C. Morais}
\author[itesm]{Pilar Arroyo-L\'opez}



\cortext[cor1]{Corresponding author at: xjavierpereira7@gmail.com (J.Pereira)}

\address[utc]{Universidad Tecnol\'ogica de Chile Inacap, Santiago, Chile (xjavierpereira7@gmail.com); \\}
\address[ap]{WMG, The University of Warwick, UK (pedro.contreras@gmail.com);}
\address[ufpe]{Universidade Federal de Pernambuco, CDSID (dcmorais@cdsid.org.br);}
\address[itesm]{Tecnologico de Monterrey, Campus Toluca, M\'exico (pilar.arroyo@itesm.mx);}



\begin{abstract}
The Human Development Index (HDI) has been proposed as a mean to encourage nations to focus on people capabilities to develop a country.  HDI calculation is based on three indices: the life expectancy, the years of education and the Gross National Income. However, several studies criticize the highly compensatory nature of the computation method and seeming precision of the information sources. Thus, last decade, non-compensatory multi-criteria clustering approaches, which also consider imprecise information,  have been proposed as an alternative to the ranking process underlying the HDI calculation. In this paper, a clustering approach, integrating  the $K$-means algorithm, is presented in which uncertain parameters are considered.  To the best of our knowledge, this is the first time that a non-compensatory and stochastic multi-criteria clustering approach is applied to group countries included in the HDI Report. Results obtained with this approach are compared to the deterministic 2018 HDI ranking, finding that notable differences are detected. Limitations are discussed and future research is proposed.
\end{abstract}

\begin{keyword}
HDI \sep clustering \sep $K$-means \ \sep outranking relation 
\end{keyword}

\end{frontmatter}


\section{Introduction}

Since 1990, the United Nations Development Programme (UNDP) releases the Human Development Index (HDI) to emphasize that people and their capabilities should be the ultimate criteria for assessing the development of a country, not only economic growth. The 2018 HDI ranking process evaluates  189 United Nations countries on the basis of three criteria: the life expectancy, the years of education and the Gross National Income (GNI). In addition, based on the HDI quartiles, four ordered groups of countries can be identified: very high human development, high human development, medium human development, and low human development.  

Despite the broad use of HDI  in business, policy-making, research, development-political debates, allocation of development aid, international climate accord designs, and economics \citep{Wolff2011}, it has been object of three main criticisms \citep{Noorbakhsh1998, Berenger2007, Klugman2011, Martinez2013}: (1) the compensatory effect; (2) the underlying concepts to define the dimensions; (3) the categorization process that uses the COP quartiles.  Regarding the compensatory effect, it is easy to see that a country exhibiting a bad performance in one index, can compensate it with a good performance in another. However, this is not desirable since, for instance, bad health indices can be hidden by good GNI, which can exacerbes inequality in access to health systems.   

Non-compensatory  methods have been proposed as an alternative to the HDI calculation method. For instance, \cite{Natoli2011} analyze three aggregation techniques designed for benchmarking and ranking of countries according to aggregated dimensions: additive methods, geometric aggregations, and non-compensatory methods.  They focus on the Condorcet approach as a non-compensatory aggregation technique for progress measures.  \cite{Lozano2009} propose to use the minimum of the component indexes in HDI instead of the arithmetic average. \cite{Mazziotta2015} compare two non-compensatory composite indices for measuring multidimensional phenomena and monitoring their changes over time: a non-linear composite index and a two-parameter function, which is an intermediate case between a compensatory and a full non-compensatory index. The authors found that the non-linear composite seems to be less compensatory that the second one.

Several scholars have proposed  ordered clustering strategies to partition the set of countries included in the HDI ranking,  based on non-compensatory multi-criteria approaches \citep{DeSmet2014}.   \cite{Boujelben2016} propose an approach that integrates PROMETHEE and $K$-means, a popular greedy algorithm for partitioning a set of elements into a pre-defined number of clusters so as to minimize the distances to the cluster centroids (the cluster centers). The underlying idea of their approach is based on the notion of belief distance between the alternatives and the centroids. \cite{Chen2018} propose a PROMETHEE-based algorithm, which also integrates the $K$-means rationale, to cluster countries in the HDI Report. They found that the ordered clustering is highly consistent with the HDI ranks. \cite{DeSmet2012} propose an exact multi-criteria algorithm to find HDI ordered categories, based on valued preference degrees.  They show the consistency between the obtained ordered partition and the HDI ranking.  \cite{Monteiro2018} propose a supervised classification approach based on the ELECTRE TRI method in which the HDI categories are defined \emph{a priori} by using fixed category boundaries. Different from other proposals using the outranking-based models, these authors consider that there is no imprecise data in HDI information sources. 

However, two types of imprecision have been highlighted by authors analyzing the quality of information used in the HDI calculation.  Firstly, the weighting process assigning equal importance levels to dimensions in HDI has been criticized and methods to deal with have been proposed \citep{Zheng2015}. In this article, we are not considering this kind of imprecision.  Secondly, the importance of considering imprecise data in constructing composite indicators has been remarked \citep{Cherchye2011}.  There are multiple indicators used for different kind of indexes, for instance, the Environmental Performance Index, the Internal Market Index, the Human Development Index, or the Technology Achievement Index. However, for some indicators only interval information is available, within what a true value is believed to lie.   \cite{Wolff2011} analyzed three sources of data error in HDI calculation and classification: measurement error due to data revisions, data error due to formula updating and misclassification due to inconsistent categories cut-off values. Actually,  measurement error from data updating impacts the rough data, the indices normalization and the HDI calculation. They found that from 11\% up to 34\% of all countries could be interpreted as misclassified in the development categories. These authors also found that on average the expected absolute deviation is nine rank positions. These results mean that interpretation of HDI ranks must be very careful.  Moreover, methods applied to HDI, which are based on the outranking relation, as for instance those using the PROMETHEE or the ELECTRE TRI methods, consider stable parameters.  However, this choice seems to be inadequate since the imprecision introduced by data updating and   inconsistent categories cut-off values.  Thus, parameter values could be supposed to be uncertain, something that has not been yet considered in the HDI clustering literature.

Essentially, clustering processes integrating $K$-means lie in two features \citep{DeSmet2009,Lolli2014,Panapakidis2018, Chen2018}. First, the definition of  an indifference-based metric that measures how indifferent is an alternative to a central alternative, or centroid, representing a category. Thus, an alternative is assigned to the cluster where the most indifferent centroid is found. Second, once all the alternatives have been assigned, the centroids are updated, using the information regarding the alternatives in their respective groups.  The process continues until a stop condition is satisfied. 

Although the $K$-means and the outranking-based methods can be used to find the same number of country clusters proposed by the ranking-based HDI methodology, these use compensatory net flow aggregation processes to obtain global preference relations. Thus, the methods do not necessarily guarantee the separability between any pair of central actions representing two different categories, a desired property of preference-based classification processes \citep{roy2012}. This means that a strict preference relation between any pair of centroids at each criterion level is not necessarily satisfied. As a consequence, supplementary procedures must be defined to order the clusters.  In these methods,  an alternative must be compared to each other, at least once. This information is stored in the form of a $n \times n$ matrix which must be traversed through on each iteration of the clustering algorithm. These are $O(IKn^2) \sim O(n^2)$ processes, where $I, K$ are the number of iterations of the algorithm (before finding a solution) and the number of clusters, respectively. We argue that a $O(Kn)$ method may be implemented for the assignment process by simply comparing each country at most to the $K$ centroids in the model.  Actually, we propose the usage of the ELECTRE TRI-C sorting method rationale to assign alternatives, in which central reference alternatives are used to define the categories, but updating these centroids by applying the $K$-means approach.

In this article, a stochastic approach for ordered clustering in the HDI problem is proposed, which integrates an outranking-based classification process with $K$-means, such that:  (1) building ordered clusters is an integral part of the procedure, without need for supplementary processes; an assignment rule proposed in the ELECTRE TRI-C sorting method \citep{almeida2008,Almeida2012} is used in the process, extending rules proposed in sorting methods to clustering; (2) uncertain thresholds are considered such that robustness analysis is performed. Thus, the proposed approach integrates  a Monte Carlo Simulation process in which sets of  parameter values are passed as input to the clustering algorithm. To the best of our knowledge, this is the first approach that deals with these two problems.

The article is organized as follows. In Section \ref{notation} basics of the clustering approach are presented. Section \ref{approach} presents the clustering approach. In Section \ref{application}  countries included in the HDI 2018 report are clustered by following the proposed approach.  Results are presented and discussed.  Limitations of this approach and final conclusions are described in Section \ref{conclusions}.

%\section{Human Development Index}
%
%Since the first publication in 1990, the Human Development Index has been used as a benchmark to encourage development of nations and to foster policy makers to focus on three dimensions:  long and healthy life, access to knowledge and a decent standard of living.  The HDI is the geometric mean of normalized indices for each of the three dimensions, constructed from four indicators \citep{UNDP2019}: Life expectancy at birth (LE); Expected years of schooling (EYS); Mean years of schooling (MYS); and Gross National Income per capita (GNIpc) .  
%
%In order to calculate the HDI, two steps are applied: (1) Creating the dimension indices; (2) Aggregating the dimension indices into the HDI.  Step (1) consists of normalizing the four mentioned indicators according to the following formula:
%
%\begin{equation}
%\mbox{Dimension index}=\frac{\mbox{indicator value - minimum value}}{\mbox{maximum value - minimum value}}. \label{index}
%\end{equation}
%
%\noindent
%In the case of the two normalized education indices, the arithmetic mean is taken.  For the GNI, the natural logarithm of the indicator, the minimum and the maximum values is taken. Next, in Step (2), the geometric mean of indices is taken, as follows:
%
%\begin{equation}
%HDI= (I_{Health} \cdot I_{Education} \cdot I_{Income})^{1/3}. \label{HDI}
%\end{equation}
%
%\noindent
%It is worth to note that in this formula the same weights for all dimensions are given.   Based on the 2014 HDI, four categories of human development achievements are defined, using fixed cutoff points (COP) calculated using the quartiles ($q$) from the distributions of the component indicators averaged over 2004–2013:
%
%\begin{equation}
%COP_q= HDI(LE_q,EYS_q,MYS_q,GNIpc_q), \, q=1,2,3. \label{COP}
%\end{equation}
%
%\noindent
%These quartiles are used in the 2019's Report, as follows:
%
%
%\begin{table}[hbtp]
%\centering
%\begin{tabular}{ll}
%Category						&Limits \\
%\hline
%Very high human development 		& 0.800 and above \\
%\hline
%High human development			& 0.700-0.799		\\
%\hline
%Medium human development		& 0.550-0.699		\\
%\hline
%Low human development			& Below 0.550\\
%\hline
%\end{tabular}
%\end{table}
%
%
%%\subsection{HDI clustering problem}
%
%Three main criticisms to HDI have been reported by different authors \citep{Noorbakhsh1998, Berenger2007, Klugman2011, Martinez2013}: (1) the compensatory effect; (2) the underlying concepts to define the dimensions; (3) the categorization process that uses the COP quartiles.  Regarding the compensatory effect, it is easy to see that a country exhibiting a bad performance in one index, can compensate it with a good performance in another. However, this is not desirable since, for instance, bad health indices can be hidden by good GNI, which can exacerbes inequality in access to health systems.   
%
%Non-compensatory  methods have been proposed as an alternative to the HDI calculation method. For instance, \cite{Natoli2011} analyze three aggregation techniques designed for benchmarking and ranking of countries according to aggregated dimensions: additive methods, geometric aggregations, and non-compensatory methods.  They focus on the Condorcet approach as a non-compensatory aggregation technique for progress measures.  \cite{Lozano2009} propose to use the minimum of the component indexes in HDI instead of the arithmetic average. \cite{Mazziotta2015} compare two non-compensatory composite indices for measuring multidimensional phenomena and monitoring their changes over time: a non-linear composite index and a two-parameter function, which is an intermediate case between a compensatory and a full non-compensatory index. The authors found that the non-linear composite seems to be less compensatory that the second one.
%
%
%Non-compensatory  methods have been proposed to classify countries in the HDI Report.   \cite{Boujelben2016} propose an approach that integrates PROMETHEE and $K$-means. The underlying idea of their approach is based on the notion of belief distance between the alternatives and the centroids. \cite{Chen2018} propose a PROMETHEE-based algorithm, which integrates the $K$-means rationale, to cluster countries in the HDI Report. They found that the ordered clustering is highly consistent with the HDI ranks. \cite{DeSmet2012} propose an exact multi-criteria algorithm to find HDI ordered categories, based on valued preference degrees.  They show the consistency between the obtained ordered partition and the HDI ranking.  \cite{Monteiro2018} propose a supervised classification approach based on the ELECTRE TRI method in which the HDI categories are defined \emph{a priori} by using fixed category boundaries. Different from other proposals using the outranking-based models, these authors consider that $p=q=0$, that is, there is no imprecision. 
%
%Two types of imprecision have been highlighted by authors analyzing the quality of information used in the HDI calculation.  Firstly, the weighting process assigning equal importance levels to dimensions in HDI has been criticized and methods to deal with have been proposed \citep{Zheng2015}. In this article, we are not considering this kind of imprecision.  Secondly, the importance of considering imprecise data in constructing composite indicators has been remarked \citep{Cherchye2011}.  There are multiple indicators used for different kind of indexes, for instance, the Environmental Performance Index, the Internal Market Index, the Human Development Index, or the Technology Achievement Index. However, for some indicators only interval information is available, within what a true value is believed to lie.   \cite{Wolff2011} analyzed three sources of data error in HDI calculation and classification: measurement error due to data revisions, data error due to formula updating and misclassification due to inconsistent categories cut-off values. Actually,  measurement error from data updating impacts the rough data, the indices normalization and the HDI calculation. They found that from 11\% up to 34\% of all countries could be interpreted as misclassified in the development categories. These authors also found that on average the expected absolute deviation is nine rank positions. These results mean that interpretation of HDI ranks must be very careful.   
%
%
%
%Thus, several scholars have proposed  ordered clustering strategies to find a partition of countries, which we call the \emph{HDI problem}, based on non-compensatory multi-criteria approaches \citep{DeSmet2014,Boujelben2016}.  Most of non-compensatory approaches for computing the HDI clustering use the global outranking relation to model imprecise information regarding the evaluation of alternatives. Outranking was early proposed in Elimination Et Choix Traidusant la Realite (ELECTRE) methods \citep{figueira2010} and further became an essential feature in  Preference Ranking Organization METHod for Enrichment of Evaluations (PROMETHEE) methods \cite{brans85}. The outranking relation helps to model global  assertions from kind ``the alternative $b$ is at least as good as the alternative $a$'', based on information collected at the level of each criteria. 
%
%Some of the new MCDA clustering approaches to HDI integrate the $K$-means rationale \citep{DeSmet2009,Lolli2014,Panapakidis2018, Chen2018}, a popular greedy algorithm for partitioning a set of elements into a pre-defined number of clusters so as to minimize the distances to the cluster centroids (the cluster centers). Essentially, these clustering processes lie in two features of $K$-means. First, the definition of  an indifference-based metric measures how indifferent is an alternative to a centroid. Thus, an alternative is assigned to the cluster where the most indifferent centroid is found. Second, once all the alternatives have been assigned, the centroids are updated, using the information regarding the alternatives in their respective groups.  The process continues until a stop condition is satisfied. 
%
%Although the the $K$-means and outranking-based methods can be used to find the same number of country clusters proposed by the ranking-based HDI methodology, these use compensatory net flow aggregation processes to obtain global preference relations. Thus, the methods do not necessarily guarantee the separability between any pair of central actions representing two different categories, a desired property of preference-based classification processes \citep{roy2012}. This means that a strict preference relation between any pair of centroids at each criterion level is not necessarily satisfied. As a consequence, supplementary procedures must be defined to order the clusters.  In these methods,  an alternative must be compared to each other, at least once. This information is stored in the form of a $n \times n$ matrix which must be traversed through on each iteration of the clustering algorithm. These are $O(IKn^2) \sim O(n^2)$ processes, where $I, K$ are the number of iterations of the algorithm (before finding a solution) and the number of clusters, respectively. We argue that a $O(Kn)$ method may be implemented for the assignment process by simply comparing each country at most to the $K$ centroids in the model.  
%
%Methods based in the outranking relation need to model imprecise information by using two kind of parameters, defined for each criterion \citep{figueira2010}: the indifference and the preference thresholds. Most of time, these parameters are defined  with the decision-maker (DM) or expert judgement assistance. However, in the case of information regarding the HDI analysis, there is not an identified DM or even expert judgement can be variable because sources of information to evaluate countries in the three criteria is uncertain.  Thus, in this problem, considering single values does not seem appropriate.   
%
%
%In this article, we are going to model imprecision by the indifference and preference thresholds. However, differently from other outranking-based approaches, the parameter values are supposed to be realizations of  stochastic variables. Thus, the proposed approach is integrated  to a Monte Carlo Simulation process in which sets of  parameter values are passed as input to the ordered clustering algorithm.  In which follows, this procedure is explained in detail.



\section{Notation}\label{notation}

\subsection{Outranking relation}\label{classification}

Most of non-compensatory approaches for computing the HDI clustering use the global outranking relation to model imprecise information regarding the evaluation of alternatives. Outranking was early proposed in Elimination Et Choix Traidusant la Realite (ELECTRE) methods \citep{figueira2010} and further became an essential feature in  Preference Ranking Organization METHod for Enrichment of Evaluations (PROMETHEE) methods \cite{brans85}. The outranking relation helps to model global  assertions from kind ``the alternative $b$ is at least as good as the alternative $a$'', based on information collected at the level of each criteria. 

Non-compensatory methods based on the outranking relation could lead to higher computational complexity. However, problems exist in which compensation among criteria is not allowable. We place the present article in that kind of situations and propose a clustering model based on the outranking relation, as defined in the ELECTRE family of methods \citep{figueira2010}.

Let $A=\{a_1,a_2,\ldots,a_n\}$ be a set of actions (alternatives);  let $F=\{g_1,g_2,\ldots,g_m\}$ be a coherent family of criteria used to evaluate each action and $W=\{w_j \mid 0 \leq w_j \leq 1, \sum_{j=1}^m w_j =1\}$ the criteria weights.  The vector of performances of an action  $(g_1(a),g_2(a),\ldots,g_m(a))$   is called the \emph{evaluation profile}. 



The outranking relation, interpreted as the assertion ``$b$ is at least as good as $a$'',  can be measured on each criterion in terms of  the \emph{partial direct concordance index}, defined as follows: 

{\footnotesize
\begin{equation}
c_j(b,a) =
\begin{cases}
	0							& \mbox{if  $g_j(a)-g_j(b) > p_j$}, \\
	1							& \mbox{if  $g_j(a)-g_j(b) \leq q_j$},   \\
	\dfrac{g_j(b)-g_j(a)+p_j}{p_j-q_j} 	& \mbox{otherwise}, 				     	      
 \end{cases} 
 \label{credibility}
 \end{equation} 
}

\noindent
where $p_j,q_j$ are two imprecision parameters called direct preference threshold and  direct indifference threshold, respectively.  Similarly,  let us consider the assertion ``$a$ is at least as good as $b$'', by using inverse thresholds $p'_j,q'_j$, such that a \emph{partial inverse concordance index} may be defined as 

{\fontsize{7}{7}
\begin{equation}
c^{inv}_j(a,b) =
\begin{cases}
	0							& \mbox{if  $g_j(b)-g_j(a) > p'_j$}, \\
	1							& \mbox{if  $g_j(b)-g_j(a) \leq q'_j$},   \\
	\dfrac{g_j(a)-g_j(b)+p'_j}{p'_j-q'_j} 	& \mbox{otherwise}. 				     	      
 \end{cases} 
 \label{invcredibility}
 \end{equation} 
}


\noindent
A detailed explanation regarding the properties that direct and inverse thresholds must satisfy can be found in \cite{roy2012}.  The partial direct and inverse concordance indices allow to compute the global concordance indices as follows:

\begin{eqnarray}
\sigma_D(b_h,a) 	&=& \sum_{j=1}^{m} w_j c_j(b_h,a),\label{generalsigmaDG}\\
\sigma_I(a,b_h) 	&=& \sum_{j=1}^{m} w_j c^{inv}_j(a,b_h). \label{generalsigmaIG}
\end{eqnarray} 


\subsection{Outranking-based classification}\label{outrankingclass}

Let $C_h, (h=1,\ldots,H; H\geq 2)$ be a set of ordered categories such that $C_H \succ C_{H-1}\succ \ldots \succ C_1$, where  the relation $\succ$ means that for any $C_i, C_ j $ such that $i>j$, an element $a_i \in C_i$ is not worse than any element $a_j \in C_j$. Conversely, $a_j$ is worse than $a_i$, in terms of preferences.   A special set of actions $B=\{b_1,\ldots, b_H\}$ is defined, in which $b_h$ represents the \emph{centroid} of the category $C_h$. 



\def \q {1}
\def \p {2}
\def \qq {1.5}
\def \pp {3}
\def \xmina{-3}
\def \xmaxa{.5}
\def \xminb{-.5}
\def \xmaxb{4}
\def \ymax{1}
\def \xb{2*\p}

\begin{figure}[hbtp]
\scriptsize
\centering
\begin{tikzpicture}
\begin{axis}[
		clip=false,
		xmin=0,xmax=5,
		ymin=0,ymax=1,
		height=4cm,
		width=7cm,
		axis line style={draw=none}, 
		xticklabels=\empty,	
		xtick=\empty,
		xtick={0},	
		ytick=\empty,
		xtick pos=left,
		xtick distance=1,
		every axis x label/.style={at={(current axis.right of origin)},anchor=west}
		]
		\addplot [domain=-\p:-\q,smooth, thick] {(x+\p)/(\p-\q)};
		\addplot [domain=\xmina:-\p,smooth, thick] {0};
		\addplot [domain=-\q:\qq,smooth, thick] {1};
		\addplot [smooth,dashed]  coordinates {(-\q,0)(-\q,\ymax)} node{} ;
		\addplot [smooth,->]  coordinates {(\xmina,0)(\xmina,\ymax+.5)} node[above]{$\mu_j(a,b_h)$} ;
		\addplot [domain=\pp:\xmaxb,smooth, thick] {0};
		\addplot [domain=-\p:\xmaxb,smooth,thin,->] {0};
		\addplot [domain=\qq:\pp,smooth, thick] {(-x+\pp)/(\pp-\qq)};
		\addplot [smooth,dashed]  coordinates {(\qq,0)(\qq,\ymax)} node{} ;
    		\node  at (axis cs:\qq-.1,-0.1) {\tiny$10+q_j$};
    		\node  at (axis cs:\pp,-0.1) {\tiny$10+p_j$};
    		\node  at (axis cs:-\p-.3,-0.1) {\tiny$10-p_j'$};
    		\node  at (axis cs:-\q-.1,-0.1) {\tiny$10-q_j'$};
    		\node  at (axis cs:0,-0.1) {\tiny$10$};
		\node  at (axis cs:4.5,0) {${\tiny g_j(a)}$};
\end{axis}
\end{tikzpicture}
\caption{Membership function in continuous and discrete cases}
\label{mu}
\end{figure}


Let us define the following trapezoidal fuzzy number (TrFN)  \citep{Ban2011}:

\begin{equation}
\mu_j(a,b_h)=\min\{c_j(b_h,a), c^{inv}_j(a,b_h)\},
\label{trfnmu}
\end{equation}


\noindent
which can be interpreted as a fuzzy indifference  relation, constructed from two outranking relations \citep{perny1992}. In Figure \ref{mu},  an example is shown where $b_h$, with $g_j(b_h)=10$ on a criterion $j$,  is the centroid of a class $C_h$. Thus, any alternative $a$ can be evaluated as belonging to that class or not, by using the membership function $\mu_j(a,b_h)$.

Conversely, deconstructing $\mu_j(a,b_h)$ into the two outranking relations $c_j(b_h,a)$ and  $c^{inv}_j(a,b_h)$, the credibility indices \eqref{generalsigmaDG} and \eqref{generalsigmaIG} may be calculated and the following assignment rules can be defined \citep{pereira2018}:

\begin{enumerate}
\item
\emph{Descending Rule}. Let $\lambda \in [0.5,1]$ be a minimum credibility level. Decrease $h$ from $H+1$ until the first $t$ such that $\sigma_I(a,b_t)\geq \lambda$:

\begin{enumerate}
\item
If $t=H+1$, assign $a$ to $C_H$.
\item
If $t=0$, assign $a$ to $C_1$.
\item
For $0<t<H+1$, 

if $\min\{\sigma_I(a,b_t),\sigma_D(b_t,a)\} > \min\{\sigma_I(a,b_{t+1}),\sigma_D(b_{t+1},a)\}$ then assign $a$ to $C_t$; otherwise, assign $a$ to $C_{t+1}$.
\end{enumerate} 

\item
\emph{Ascending Rule}. Let $\lambda \in [0.5,1]$ be a minimum credibility level. Increase $h$ from $0$ until the first $t$ such that $\sigma_D(b_t,a)\geq \lambda$:

\begin{enumerate}
\item
If $t=1$, assign $a$ to $C_1$.
\item
If $t=H+1$, assign $a$ to $C_H$.
\item
For $0<t<H+1$, 

if $\min\{\sigma_I(a,b_t),\sigma_D(b_t,a)\} > \min\{\sigma_I(a,b_{t-1}),\sigma_D(b_{t-1},a)\}$ then assign $a$ to $C_t$; otherwise, assign $a$ to $C_{t-1}$.
\end{enumerate} 
\end{enumerate}


Essentially, the descending and the ascending rules evaluate how indifferent an alternative $a$ is to every $b_h$.  Thus, $a$ is assigned to the class for which it is closest.  In practice, these rules may assign an alternative to different classes. Therefore, in a particular application of the algorithm presented below, just one of them must be chosen.


\subsection{$K$-means algorithm}

$K$-means is a classical unsupervised method to partitioning a set of elements into a set of disjoint homogeneous groups, by minimizing the following objective function 

\begin{equation}
\min \sum_{h=1}^{K} \sum_{i=1}^{n} I_{ih} \left\| a_i-b_h \right\|,
\end{equation}


\noindent
where  $I_{ih}=1$ if $a_i \in C_h$, $0$ otherwise; and $ \left\| a_i-b_h \right\|$ is the Euclidean distance between $a_i$ and $b_h$.  Classical $K$-means algorithm consists of three steps: (1) define $A, K, b_h$ as inputs; (2) assign elements of $A$ into clusters; (3) update the cluster centers and return to step (2), until a stop condition is satisfied.   

Step (2) assigns an element $a_i$ to the group where the Euclidean distance between the element and the respective cluster center is minimized.  Step (3) updates the cluster centroids as follows


\begin{equation}
g_j(b_h) = \frac{1}{\mid C_h \mid} \sum_{a_i \in C_h} g_j(a_i) \quad j=1,\ldots,m. \label{newcentroid}
\end{equation}

\noindent
Thus, the algorithm is iterated a number of times, or until the  change of centroids is less or equal than a given tolerance.


\section{Clustering approach}\label{approach}

The approach proposed here assumes that $K\geq 2$ ordered clusters must be found: $C_K \succ \ldots \succ C_1$. Similarly to the original $K$-means algorithm, an iterative constructive clustering process is performed until a stop condition is satisfied, as described  in Figure \ref{clustering}.  




\begin{figure}[hbtp]
\begin{center}
\fontsize{7}{7}{
\tikzstyle{bigbox}=[align=center,rectangle, draw=black, rounded corners,  anchor=north, minimum width=3.5cm, text width=3cm, minimum height=.7cm, node distance=.4cm]
\tikzstyle{myarrow}=[->, >= triangle 45]
\begin{tikzpicture}[
	scale=0.7,
	blueb/.style={
  		draw=black,
¡  		rounded corners,
  		text width=2cm,
  		font={\sffamily\color{black}}}
		]
    \node (defafw)[bigbox]
        {
            	1. Define $A, F, W, K, s=0$
        };
    \node (defpq)[bigbox,below= of defafw]
        {
            	2. Define $p_j,p'_j,q_j,q'_j,s=s+1$
        };
    \node (initcentroids)[bigbox,below= of defpq]
        {
            	3. Define $t=0, b_h^{(0)}$
        };
    \node (indices) [bigbox,   below=  of initcentroids]
        {
            	4. Compute $\sigma_D, \sigma_I$
	};
    \node (assign) [bigbox,   below=  of indices]
        {
            	5. Assign alternatives
	};
    \node (centroid) [bigbox,   below= of assign]
        {
            	6. Compute $t=t+1$ and  $b_h^{(t)}$ 
	};
   \node (decision) [draw, diamond, align=center, minimum width=2.5cm, minimum height=2.5cm, aspect=2, inner sep=-3pt, below=.4cm of centroid] 
        {
        $\mid b_h^{(t)} -b_h^{(t-1)} \mid < \epsilon$,\\
        or\\
	$t=Maxit$
        };
   
   \node (stochastic) [draw, diamond, align=center,  minimum width=3.0cm, minimum height=2.5cm, aspect=2, inner sep=-3pt, below=.4cm of decision] 
        {
	$s=Maxst$
        };

    \node (final) [bigbox,   below=of stochastic]
        {7. Compute frequency indices
	};
       
   \draw[myarrow] (defafw.south) -> (defpq.north);
   \draw[myarrow] (defpq.south) -> (initcentroids.north);
   \draw[myarrow] (initcentroids.south) -> (indices.north);
   \draw[myarrow] (indices.south) -> (assign.north);
   \draw[myarrow] (assign.south) -> (centroid.north);
  \draw[myarrow] (centroid.south) -> (decision.north);
  \draw[myarrow] (decision.south) node[left] {yes} -> (stochastic.north);
  \draw[myarrow] (decision.west) |- node[above] {no} ++(-17mm,0) --  ($(indices.west) + (-15mm,0)$) -> (indices.west);
  \draw[myarrow] (stochastic.south) node[left] {yes} -> (final.north);
  \draw[myarrow] (stochastic.west) |- node[above] {no} ++(-30mm,0) --  ($(defpq.west) + (-27mm,0)$) -> (defpq.west);

\end{tikzpicture}
}
\caption{Outranking-based approach to ordered clustering}
\label{clustering}   
\end{center}
\end{figure}

Initially, the set of alternatives, the family of criteria, the criteria weights, and the number of clusters are identified.  Next, the imprecision parameters are set. Stochastic preference and indifference thresholds are considered in this approach. Thus, each time that the Step 2 is performed, a new set of values for $p_j,q_j,p'_j,q'_j \,(j=1,\ldots,m)$ are generated and used as input to the Step 3, where the new initial centroids are defined. 

%Centroids  must satisfy the strict separability condition \citep{roy2012}:  for any pair $b_h^{(0)}, b_k^{(0)}$ such that $h>k$, it follows 
%
%\begin{equation}
%b_h^{(0)} \,P\, b_k^{(0)} \Leftrightarrow g_j(b_h^{(0)})-p'_j \geq g_j(b_k^{(0)})+p_j, \forall j.
%\end{equation}
%
%\noindent
%where $P$ is a binary relation representing a strong preference.  This condition makes that fuzzy sets representing the membership functions of every category do not overlap. 


Step 4 consists of computing the credibility indices, using the expressions \eqref{generalsigmaDG} and  \eqref{generalsigmaIG}. In Step 5, the alternatives are assigned into the classes, using the descending (or the ascending) rule proposed in Section \ref{outrankingclass}.

In Step 6, the evaluation profile of each new centroid in the iteration $t$ and class $C_h$ is computed as follows:

\begin{equation}
g_j(b_h^{(t)}) = \frac{1}{\mid C_h^{t,j} \mid} \sum_{a_i \in C_h^{t,j}} g_j(a_i) \quad j=1,\ldots,m,\label{newcentroid}
\end{equation}

\noindent
where $C_h^{t,j}=\{a_i \in C_h \mid g_j(b_h^{(t)})-p'_j\leq g_j(a_i) \leq g_j(b_h^{(t)})+p_j\}$.  Whenever $C_h^{t,j}=\emptyset$, then $g_j(b_h^{(t)})=g_j(b_h^{(t-1)})$.   


\cite{roy2012} propose that a \emph{separability condition} must be satisfied in classification problems in which a set of central reference alternatives is used: any two central reference alternatives must be separated enough such that the strong preference property holds, at the level of each criterion. We extend this property to the cases in which the direct and the inverse preference thresholds are used:

\begin{cond}
\emph{The set of reference alternatives $B$ fulfills the strict separability condition if and only if}

\begin{equation}
g_j(b_{h-1})+p_j \leq g_j(b_h)-p'_j \,,(h=2,\ldots,H; j=1,\ldots,n). \label{separability}
\end{equation}

\end{cond}

\noindent
If this condition is hold along the iterative process, the resulting cluster centers are ordered. The approach proposed satisfies this property.


In this approach, thresholds are stochastic variables where values are generated by a Monte Carlo Simulation process. Thus, the whole process is repeated $Maxst$ times, once for each set of stochastic parameter values.  Given a set of parameter values, the inner loop in Figure \ref{clustering} iterates up to one out of two stop conditions applies: 1) the change of evaluation profiles is  lower than a pre-defined $\epsilon$ value, or 2) the maximum number of iterations, $Maxit$,  is reached.   

In Step 7, the number of times that every alternative is assigned into each category is computed. Thus, a frequency profile is obtained for each alternative.   

In the following section, this approach is applied to the HDI problem.





\section{Clustering HDI countries}\label{application}

Data published in the 2018 HDI Report is retrieved to perform the analysis. There are 189 countries to be clustered. The family of criteria to be considered consists of the three indices: life expectancy ($g_1$), years of schooling ($g_2$) and GNI  ($g_3$).  These indices are computed using the formulae defined in the HDI Technical Notes \citep{UNDP2019}.   As proposed by UNDP, weights are considered to be equal to $1/3$ for each criterion. 

\subsection{Initial settings}

In order to have an idea about the number of clusters, the $K$-means Elbow Method was run on indices computed from HDI data. The Elbow Method is used to determine the optimal number of clusters in $K$-means clustering \citep{Ketchen1996}. This gives four clusters,    i.e. $K=4$, which parallels the number of groups defined in the Human Development Report. In order to define an initial set of centroids, the $K$-means algorithm is run.   The following initial centroids are found, which are used as input to our approach:

\begin{eqnarray}
b_1&=&(0.27, 0.25, 0.43), \nonumber\\
b_2&=&(0.51, 0.39, 0.64), \nonumber\\
b_3&=&(0.70, 0.53, 0.83), \nonumber\\
b_4&=&(0.90, 0.71, 0.93). \nonumber
\end{eqnarray}

\noindent
For each criterion $j$, the minimum distance between consecutive centroids $m_j=\min_{h} \{g_j(b_h)-g_j(b_{h-1})\}$ is calculated, then $m_1=0.19, m_2=0.14, m_3=0.10$. Note that holding the separability condition means  $g_j(b_{h-1})+p_j \leq g_j(b_h)-p'_j$, or equivalently $g_j(b_{h-1})+p_j+p'_j \leq g_j(b_h)$. Thus, to satisfy the strict separability condition, the following restriction must be defined

\begin{equation}
p'_j+p_j \leq m_j \, (j=1,2,3). \label{thresholds}
\end{equation}

%\noindent
%We suppose that $p_j,q_j,p'_j,q'_j$ do not depend on the position where an alternative is evaluated on the criterion $j$. There are situations in which that assumption may not be valid and more complete constraints should be considered. However, in this article we assume that the restriction \eqref{thresholds} is adequate.


Let us assume that  $p_j,q_j,p'_j,q'_j$ are uncertain.  We may assume that stochastic variables $\xi_{p_j},\xi_{q_j},\xi_{p'_j},\xi_{q'_j}$ correctly model each parameter. We assume that each stochastic variable is uniformly distributed in an interval  $[\xi_m, \xi_M]$. If a stochastic interval is defined for each variable, as $[m_j-0.1, m_j+0.1] \,(j=1,2,3)$, then the following linear equations system may be defined:

\begin{eqnarray}
p_1,p'_1 &\in& [0.090, 0.290], \nonumber\\
p_2,p'_2 &\in& [0.040, 0.240], \nonumber\\
p_3,p'_3 &\in& [0.000, 0.200], \nonumber\\
q_1,q'_1 &\in& [0.045, 0.145], \nonumber\\
q_2,q'_2 &\in& [0.020, 0.120], \nonumber\\
q_3,q'_3 &\in& [0.000, 0.100], \nonumber\\
p_1+p'_1 &\leq& 0.19, \nonumber\\
p_2+p'_2 &\leq& 0.14,  \nonumber\\
p_3+p'_3 &\leq& 0.10. \nonumber
\end{eqnarray}


\noindent
A set of 3000 random values is generated for each parameter. Consequently, $Maxst=3000$.  The number of iterations is defined as $Maxit=30$ because preliminary simulations showed that this is enough to reach stable centroids. The proposed approach is implemented in Python.  

\subsection{Results}

In Table \ref{results}, four groups of alternatives  are presented, the same defined in the HDI ranking.  Thus, for instance,  countries ranging from  1 (Norway) up to 62 (Seychelles) belong to the group of very high developed countries. Clusters built by the proposed approach are coded as $C_1, C_2, C_3, C_4$ and satisfy $b_1 \prec  b_2 \prec b_3 \prec b_4$.  




\newcommand\items{4}   %Number of classes
\arrayrulecolor{white} %Table line colors

\begin{table}[hbtp]
\caption{Ordered clustering  vs HDI development groups}
\label{results}
\tiny
\hskip-4.0cm
\begin{tabular}{llll}
\begin{tabular}[t]{c*{\items}{|E}l}
\multicolumn{5}{c}{VERY HIGH DEVELOPMENT}\\\hline 
\multicolumn{1}{c}{Country} & 
\multicolumn{1}{c}{$C_1$} & 
\multicolumn{1}{c}{$C_2$} & 
\multicolumn{1}{c}{$C_3$} & 
\multicolumn{1}{c}{$C_4$} 
\\\hline
1	&	0.00 	&	0.00 	&	0.01 	&	0.99 	 \\\hline
2	&	0.00 	&	0.00 	&	0.00 	&	1.00 	 \\\hline
3	&	0.00 	&	0.00 	&	0.02 	&	0.98 	 \\\hline
4	&	0.00 	&	0.00 	&	0.00 	&	1.00 	 \\\hline
5	&	0.00 	&	0.00 	&	0.06 	&	0.94 	 \\\hline
6	&	0.00 	&	0.00 	&	0.00 	&	1.00 	 \\\hline
7	&	0.00 	&	0.00 	&	0.01 	&	0.99 	 \\\hline
8	&	0.00 	&	0.00 	&	0.02 	&	0.98 	 \\\hline
9	&	0.00 	&	0.02 	&	0.13 	&	0.86 	 \\\hline
10	&	0.00 	&	0.00 	&	0.04 	&	0.96 	 \\\hline
11	&	0.00 	&	0.00 	&	0.01 	&	0.99 	 \\\hline
12	&	0.00 	&	0.00 	&	0.02 	&	0.98 	 \\\hline
13	&	0.00 	&	0.00 	&	0.00 	&	1.00 	 \\\hline
14	&	0.00 	&	0.00 	&	0.00 	&	1.00 	 \\\hline
15	&	0.00 	&	0.00 	&	0.00 	&	1.00 	 \\\hline
16	&	0.00 	&	0.00 	&	0.00 	&	1.00 	 \\\hline
17	&	0.00 	&	0.01 	&	0.09 	&	0.90 	 \\\hline
18	&	0.00 	&	0.00 	&	0.16 	&	0.84 	 \\\hline
19	&	0.00 	&	0.00 	&	0.09 	&	0.91 	 \\\hline
20	&	0.00 	&	0.00 	&	0.01 	&	0.99 	 \\\hline
21	&	0.00 	&	0.00 	&	0.37 	&	0.63 	 \\\hline
22	&	0.00 	&	0.00 	&	0.00 	&	1.00 	 \\\hline
23	&	0.00 	&	0.00 	&	0.04 	&	0.96 	 \\\hline
24	&	0.00 	&	0.00 	&	0.03 	&	0.97 	 \\\hline
25	&	0.01 	&	0.13 	&	0.25 	&	0.61 	 \\\hline
26	&	0.00 	&	0.00 	&	0.00 	&	1.00 	 \\\hline
27	&	0.00 	&	0.02 	&	0.26 	&	0.72 	 \\\hline
28	&	0.00 	&	0.03 	&	0.15 	&	0.82 	 \\\hline
29	&	0.00 	&	0.09 	&	0.26 	&	0.65 	 \\\hline
30	&	0.00 	&	0.00 	&	0.01 	&	0.99 	 \\\hline
31	&	0.00 	&	0.00 	&	0.27 	&	0.73 	 \\\hline
32	&	0.00 	&	0.07 	&	0.23 	&	0.70 	 \\\hline
33	&	0.00 	&	0.00 	&	0.09 	&	0.91 	 \\\hline
34	&	0.00 	&	0.00 	&	0.07 	&	0.93 	 \\\hline
35	&	0.00 	&	0.11 	&	0.85 	&	0.05 	 \\\hline
36	&	0.00 	&	0.19 	&	0.73 	&	0.08 	 \\\hline
37	&	0.01 	&	0.14 	&	0.36 	&	0.49 	 \\\hline
38	&	0.00 	&	0.00 	&	0.89 	&	0.11 	 \\\hline
39	&	0.00 	&	0.00 	&	0.14 	&	0.86 	 \\\hline
40	&	0.02 	&	0.18 	&	0.26 	&	0.55 	 \\\hline
41	&	0.03 	&	0.27 	&	0.67 	&	0.04 	 \\\hline
42	&	0.00 	&	0.07 	&	0.24 	&	0.69 	 \\\hline
43	&	0.02 	&	0.18 	&	0.74 	&	0.05 	 \\\hline
44	&	0.00 	&	0.00 	&	0.90 	&	0.09 	 \\\hline
45	&	0.02 	&	0.16 	&	0.72 	&	0.10 	 \\\hline
46	&	0.00 	&	0.02 	&	0.66 	&	0.32 	 \\\hline
47	&	0.01 	&	0.13 	&	0.79 	&	0.07 	 \\\hline
48	&	0.00 	&	0.07 	&	0.38 	&	0.55 	 \\\hline
49	&	0.00 	&	0.01 	&	0.93 	&	0.06 	 \\\hline
50	&	0.00 	&	0.00 	&	0.84 	&	0.16 	 \\\hline
51	&	0.00 	&	0.01 	&	0.89 	&	0.10 	 \\\hline
52	&	0.00 	&	0.01 	&	0.90 	&	0.09 	 \\\hline
53	&	0.00 	&	0.02 	&	0.89 	&	0.09 	 \\\hline
54	&	0.00 	&	0.05 	&	0.89 	&	0.06 	 \\\hline
55	&	0.00 	&	0.00 	&	0.62 	&	0.38 	 \\\hline
56	&	0.00 	&	0.07 	&	0.50 	&	0.43 	 \\\hline
57	&	0.12 	&	0.24 	&	0.60 	&	0.03 	 \\\hline
58	&	0.03 	&	0.21 	&	0.28 	&	0.47 	 \\\hline
59	&	0.09 	&	0.23 	&	0.25 	&	0.43 	 \\\hline
60	&	0.00 	&	0.11 	&	0.87 	&	0.02 	 \\\hline
61	&	0.00 	&	0.18 	&	0.78 	&	0.04 	 \\\hline
62	&	0.01 	&	0.14 	&	0.71 	&	0.14 	 \\\hline
\end{tabular}

&

\begin{tabular}[t]{c*{\items}{|E}|}
\multicolumn{5}{c}{HIGH DEVELOPMENT}\\\hline 
\multicolumn{1}{c}{Country} & 
\multicolumn{1}{c}{$C_1$} & 
\multicolumn{1}{c}{$C_2$} & 
\multicolumn{1}{c}{$C_3$} & 
\multicolumn{1}{c}{$C_4$} 
\\\hline
63	&	0.00 	&	0.03 	&	0.89 	&	0.08 	 \\\hline
64	&	0.00 	&	0.17 	&	0.81 	&	0.02 	 \\\hline
65	&	0.01 	&	0.11 	&	0.82 	&	0.06 	 \\\hline
66	&	0.01 	&	0.16 	&	0.76 	&	0.06 	 \\\hline
67	&	0.00 	&	0.20 	&	0.76 	&	0.04 	 \\\hline
68	&	0.03 	&	0.21 	&	0.27 	&	0.48 	 \\\hline
69	&	0.01 	&	0.10 	&	0.56 	&	0.33 	 \\\hline
70	&	0.00 	&	0.00 	&	0.74 	&	0.26 	 \\\hline
71	&	0.00 	&	0.07 	&	0.88 	&	0.05 	 \\\hline
72	&	0.00 	&	0.01 	&	0.46 	&	0.53 	 \\\hline
73	&	0.04 	&	0.27 	&	0.67 	&	0.02 	 \\\hline
74	&	0.02 	&	0.26 	&	0.68 	&	0.03 	 \\\hline
75	&	0.01 	&	0.18 	&	0.77 	&	0.04 	 \\\hline
76	&	0.04 	&	0.22 	&	0.70 	&	0.05 	 \\\hline
77	&	0.09 	&	0.23 	&	0.62 	&	0.06 	 \\\hline
78	&	0.03 	&	0.26 	&	0.64 	&	0.08 	 \\\hline
79	&	0.08 	&	0.23 	&	0.59 	&	0.10 	 \\\hline
80	&	0.05 	&	0.22 	&	0.67 	&	0.06 	 \\\hline
81	&	0.00 	&	0.06 	&	0.92 	&	0.02 	 \\\hline
82	&	0.08 	&	0.22 	&	0.64 	&	0.06 	 \\\hline
83	&	0.01 	&	0.21 	&	0.75 	&	0.03 	 \\\hline
84	&	0.02 	&	0.20 	&	0.74 	&	0.04 	 \\\hline
85	&	0.08 	&	0.25 	&	0.63 	&	0.04 	 \\\hline
86	&	0.02 	&	0.19 	&	0.72 	&	0.06 	 \\\hline
87	&	0.00 	&	0.28 	&	0.71 	&	0.01 	 \\\hline
88	&	0.00 	&	0.08 	&	0.88 	&	0.05 	 \\\hline
89	&	0.08 	&	0.23 	&	0.65 	&	0.03 	 \\\hline
90	&	0.04 	&	0.25 	&	0.67 	&	0.04 	 \\\hline
91	&	0.13 	&	0.22 	&	0.59 	&	0.06 	 \\\hline
92	&	0.00 	&	0.31 	&	0.67 	&	0.02 	 \\\hline
93	&	0.10 	&	0.38 	&	0.51 	&	0.01 	 \\\hline
94	&	0.02 	&	0.53 	&	0.45 	&	0.01 	 \\\hline
95	&	0.04 	&	0.41 	&	0.54 	&	0.02 	 \\\hline
96	&	0.01 	&	0.22 	&	0.75 	&	0.02 	 \\\hline
97	&	0.00 	&	0.41 	&	0.57 	&	0.01 	 \\\hline
98	&	0.08 	&	0.27 	&	0.60 	&	0.04 	 \\\hline
99	&	0.00 	&	0.21 	&	0.79 	&	0.00 	 \\\hline
100	&	0.05 	&	0.31 	&	0.64 	&	0.01 	 \\\hline
101	&	0.02 	&	0.49 	&	0.47 	&	0.01 	 \\\hline
102	&	0.02 	&	0.31 	&	0.66 	&	0.00 	 \\\hline
103	&	0.01 	&	0.22 	&	0.75 	&	0.02 	 \\\hline
104	&	0.16 	&	0.27 	&	0.56 	&	0.02 	 \\\hline
105	&	0.00 	&	0.16 	&	0.81 	&	0.04 	 \\\hline
106	&	0.02 	&	0.51 	&	0.46 	&	0.01 	 \\\hline
107	&	0.01 	&	0.59 	&	0.41 	&	0.00 	 \\\hline
108	&	0.06 	&	0.81 	&	0.12 	&	0.00 	 \\\hline
109	&	0.00 	&	0.54 	&	0.46 	&	0.00 	 \\\hline
110	&	0.10 	&	0.34 	&	0.54 	&	0.01 	 \\\hline
111	&	0.08 	&	0.47 	&	0.44 	&	0.02 	 \\\hline
112	&	0.00 	&	0.24 	&	0.75 	&	0.01 	 \\\hline
113	&	0.02 	&	0.70 	&	0.29 	&	0.00 	 \\\hline
114	&	0.02 	&	0.40 	&	0.55 	&	0.02 	 \\\hline
115	&	0.10 	&	0.64 	&	0.26 	&	0.00 	 \\\hline
116	&	0.12 	&	0.43 	&	0.44 	&	0.02 	 \\\hline
\end{tabular}

&

\begin{tabular}[t]{c*{\items}{|E}|}
\multicolumn{5}{c}{MEDIUM DEVELOPMENT}\\\hline 
\multicolumn{1}{c}{Country} & 
\multicolumn{1}{c}{$C_1$} & 
\multicolumn{1}{c}{$C_2$} & 
\multicolumn{1}{c}{$C_3$} & 
\multicolumn{1}{c}{$C_4$} 
\\\hline
117	&	0.00 	&	0.21 	&	0.78 	&	0.01 	 \\\hline
118	&	0.07 	&	0.28 	&	0.63 	&	0.03 	 \\\hline
119	&	0.02 	&	0.31 	&	0.65 	&	0.02 	 \\\hline
120	&	0.16 	&	0.62 	&	0.22 	&	0.00 	 \\\hline
121	&	0.29 	&	0.19 	&	0.50 	&	0.02 	 \\\hline
122	&	0.00 	&	0.38 	&	0.60 	&	0.02 	 \\\hline
123	&	0.10 	&	0.65 	&	0.25 	&	0.00 	 \\\hline
124	&	0.15 	&	0.39 	&	0.45 	&	0.00 	 \\\hline
125	&	0.02 	&	0.67 	&	0.30 	&	0.00 	 \\\hline
126	&	0.23 	&	0.42 	&	0.35 	&	0.00 	 \\\hline
127	&	0.26 	&	0.47 	&	0.27 	&	0.00 	 \\\hline
128	&	0.15 	&	0.26 	&	0.57 	&	0.01 	 \\\hline
129	&	0.19 	&	0.51 	&	0.30 	&	0.00 	 \\\hline
130	&	0.18 	&	0.69 	&	0.13 	&	0.00 	 \\\hline
131	&	0.31 	&	0.49 	&	0.20 	&	0.00 	 \\\hline
132	&	0.29 	&	0.53 	&	0.18 	&	0.00 	 \\\hline
133	&	0.12 	&	0.74 	&	0.14 	&	0.00 	 \\\hline
134	&	0.33 	&	0.54 	&	0.13 	&	0.00 	 \\\hline
135	&	0.26 	&	0.47 	&	0.27 	&	0.00 	 \\\hline
136	&	0.15 	&	0.73 	&	0.12 	&	0.00 	 \\\hline
137	&	0.18 	&	0.47 	&	0.34 	&	0.01 	 \\\hline
138	&	0.29 	&	0.61 	&	0.11 	&	0.00 	 \\\hline
139	&	0.44 	&	0.56 	&	0.00 	&	0.00 	 \\\hline
140	&	0.33 	&	0.59 	&	0.09 	&	0.00 	 \\\hline
141	&	0.18 	&	0.58 	&	0.23 	&	0.00 	 \\\hline
142	&	0.23 	&	0.67 	&	0.10 	&	0.00 	 \\\hline
143	&	0.20 	&	0.69 	&	0.11 	&	0.00 	 \\\hline
144	&	0.84 	&	0.16 	&	0.00 	&	0.00 	 \\\hline
145	&	0.39 	&	0.56 	&	0.05 	&	0.00 	 \\\hline
146	&	0.31 	&	0.52 	&	0.17 	&	0.00 	 \\\hline
147	&	0.28 	&	0.61 	&	0.11 	&	0.00 	 \\\hline
148	&	0.31 	&	0.47 	&	0.22 	&	0.00 	 \\\hline
149	&	0.41 	&	0.59 	&	0.00 	&	0.00 	 \\\hline
150	&	0.34 	&	0.66 	&	0.00 	&	0.00 	 \\\hline
151	&	0.56 	&	0.44 	&	0.00 	&	0.00 	 \\\hline
152	&	0.67 	&	0.33 	&	0.00 	&	0.00 	 \\\hline
153	&	0.33 	&	0.53 	&	0.14 	&	0.00 	 \\\hline
\end{tabular}

&

\begin{tabular}[t]{c*{\items}{|E}|}
\multicolumn{5}{c}{LOW DEVELOPMENT}\\\hline 
\multicolumn{1}{c}{Country} & 
\multicolumn{1}{c}{$C_1$} & 
\multicolumn{1}{c}{$C_2$} & 
\multicolumn{1}{c}{$C_3$} & 
\multicolumn{1}{c}{$C_4$} 
\\\hline
154	&	0.53 	&	0.40 	&	0.08 	&	0.00 	 \\\hline
155	&	0.47 	&	0.52 	&	0.01 	&	0.00 	 \\\hline
156	&	0.36 	&	0.56 	&	0.08 	&	0.00 	 \\\hline
157	&	0.32 	&	0.57 	&	0.11 	&	0.00 	 \\\hline
158	&	1.00 	&	0.00 	&	0.00 	&	0.00 	 \\\hline
159	&	0.80 	&	0.20 	&	0.00 	&	0.00 	 \\\hline
160	&	0.36 	&	0.64 	&	0.00 	&	0.00 	 \\\hline
161	&	0.68 	&	0.32 	&	0.00 	&	0.00 	 \\\hline
162	&	0.34 	&	0.59 	&	0.07 	&	0.00 	 \\\hline
163	&	0.35 	&	0.65 	&	0.00 	&	0.00 	 \\\hline
164	&	1.00 	&	0.00 	&	0.00 	&	0.00 	 \\\hline
165	&	0.85 	&	0.15 	&	0.00 	&	0.00 	 \\\hline
166	&	0.61 	&	0.39 	&	0.00 	&	0.00 	 \\\hline
167	&	0.33 	&	0.67 	&	0.00 	&	0.00 	 \\\hline
168	&	0.83 	&	0.17 	&	0.00 	&	0.00 	 \\\hline
169	&	0.52 	&	0.48 	&	0.00 	&	0.00 	 \\\hline
170	&	0.47 	&	0.53 	&	0.00 	&	0.00 	 \\\hline
171	&	0.86 	&	0.14 	&	0.00 	&	0.00 	 \\\hline
172	&	0.38 	&	0.56 	&	0.06 	&	0.00 	 \\\hline
173	&	0.71 	&	0.29 	&	0.00 	&	0.00 	 \\\hline
174	&	0.84 	&	0.16 	&	0.00 	&	0.00 	 \\\hline
175	&	0.86 	&	0.14 	&	0.00 	&	0.00 	 \\\hline
176	&	0.52 	&	0.48 	&	0.00 	&	0.00 	 \\\hline
177	&	0.66 	&	0.34 	&	0.00 	&	0.00 	 \\\hline
178	&	0.70 	&	0.30 	&	0.00 	&	0.00 	 \\\hline
179	&	0.82 	&	0.18 	&	0.00 	&	0.00 	 \\\hline
180	&	0.85 	&	0.15 	&	0.00 	&	0.00 	 \\\hline
181	&	1.00 	&	0.00 	&	0.00 	&	0.00 	 \\\hline
182	&	1.00 	&	0.00 	&	0.00 	&	0.00 	 \\\hline
183	&	1.00 	&	0.00 	&	0.00 	&	0.00 	 \\\hline
184	&	1.00 	&	0.00 	&	0.00 	&	0.00 	 \\\hline
185	&	0.48 	&	0.52 	&	0.00 	&	0.00 	 \\\hline
186	&	1.00 	&	0.00 	&	0.00 	&	0.00 	 \\\hline
187	&	1.00 	&	0.00 	&	0.00 	&	0.00 	 \\\hline
188	&	1.00 	&	0.00 	&	0.00 	&	0.00 	 \\\hline
189	&	1.00 	&	0.00 	&	0.00 	&	0.00 	 \\\hline
\end{tabular}

\end{tabular}

\end{table}


Given a row in a tabular structure, values indicate how frequently an alternative (country) is assigned into a cluster, regarding the $Maxst$ simulations.  For instance, the alternative 1 in the group VERY HIGH DEVELOPMENT, 99\% of times is assigned into $C_4$, whilst 1\%, 0\%, and 0\% of times it is assigned into $C_3, C_2, C_1$, respectively.  Colors help to recognize these frequencies: the more intense a color, the higher the frequency an alternative reaches in a cluster.  

As observed in the VERY HIGH DEVELOPMENT group, the alternatives  1 up to 34, 39 and 42   reach high frequencies in $C_4$.  The alternatives 35 up to 62 have some weak arguments to be allocated in $C_4$ or $C_3$ (except 39  and 42, as mentioned). Note that there are nine alternatives for which the arguments to be assigned into $C_4$ are strong, equalling 1.0 (i.e., 100\% of times these alternatives are assigned into this category). 

The group named HIGH DEVELOPMENT has an important number of countries that could be ambiguously assigned into $C_4, C_3$ or $C_2$.  However, four groups of alternatives could be observed:  3 countries (68, 69 and 72), having arguments to be allocated into $C_4$ or $C_3$;  about 30\% of countries  having arguments to be allocated into $C_3$ or $C_2$; about 60\% of countries  having arguments to be allocated into $C_3$; 3 countries (108, 113 and 115) allocated into $C_2$. Thus, this group is mostly representative of $C_3$.

The group named MEDIUM DEVELOPMENT has three groups of alternatives: those with arguments to be assigned into $C_3$ or $C_2$; the alternatives having arguments to be allocated into $C_2$ or $C_1$; countries allocated into $C_1$.  However, this group is mostly representative of $C_2$.

The group named LOW DEVELOPMENT shows countries with arguments to be assigned into $C_2$ or $C_1$, revealing a clear shift to the lower part of clusters. Note that ten alternatives have strong arguments to be assigned into $C_1$, i.e. scoring 1.0 in that category. This group is mostly representative of $C_1$.




\subsection{Discussion}


Other approaches for clustering based on the outranking relation have not relieved in the uncertain nature of parameters, but only in the imprecise information available to evaluate countries, thus using deterministic thresholds \citep{DeSmet2012, DeSmet2014, Fernandez2010, Chen2018}. However, there is evidence showing that imprecision is uncertain, which depends on multiple factors, changing on time. \cite{Wolff2011} found that  the higher the development status of a country, the more precise are the reported data. This seems to encompass our results: countries in the upper part of very developed countries are strongly assigned to $C_4$.  These authors also found that when many countries are close to the group cut-off thresholds (quartiles), up to 45\% of the developing countries are misclassified.  Table \ref{results} shows this phenomenon. For instance, 40\% countries in the lower part of the VERY HIGH DEVELOPMENT group  are classified in $C_3$.  

\cite{Garcia2010} studied different sources of uncertainty both in data sources and methodological choices in the HDI calculation. They found that main sources of uncertainty resulted in non biased HDI rankings. A main finding was that shifts in ranking were minor in the upper part of the VERY HIGH DEVELOPMENT and the lower part of the LOW DEVELOPMENT  groups. However, the absolute shift in ranking seemed to be minor in all levels of development, which did not compromise the robustness of the HDI, regardless of the development level.  As a consequence, the HDI ranking remains robust.  Our results somehow contradicts such a statement. There are several countries in which the frequency of each of two consecutive classes are around 25\% and 55\% (e.g., countries 37, 40, 93, 101, 124, 137, 169).  This means that ranking of those countries will be different from the HDI ranking, because can be assigned into categories different from the HDI development groups.

\cite{Garcia2010} also studied the distribution of the rankings derived from applying different weights, normalizations and an alternative functional form of life expectancy. They found that uncertainty in weights produced shift in rankings, but without significant bias regarding the HDI ranking.  In order to explore the impact of uncertain weights, Monte Carlo simulations are performed by considering $w_j \sim U(0,1) \, (j=1,2,3)$.  In Table \ref{results2} results obtained when uncertain weights and deterministic thresholds are considered:   $p_1,p'_1 =0.20;  p_2,p'_2 =0.14; p_3,p'_3 = 0.10; q_j,q'_j=p_j/2 \,(j=1,2,3)$.  In this case, more alternatives become clearly assigned into one category, instead of being ambiguously assigned into two consecutive classes. Thus, many more frequencies, as compared to Table \ref{results} outcomes,  reach 1.00 (100\% of simulations giving the same category).   This can be interpreted as a partially confirmatory evidence in favor of the \cite{Garcia2010}'s result.

Although we may affirm that four groups of development exist, we did not found that the groups reported in the HDI Report and those built by our approach coincide. It is well known that $K$-means is highly dependent on the initial set of centroids. Therefore, we cannot affirm that our clusters are strictly valid.  However, simulations reveal that uncertainty has an impact on outcomes. 

Therefore, we have found that considering uncertainty in clustering process can impact the classification of countries in the 2018 HDI Report. However, we may conclude that uncertainty in parameters used in outranking-based approaches and uncertainty in weights do not impact in the same way. Uncertainty in parameters induce more uncertainty in outcomes.  To the best of our knowledge, this kind of uncertainty has not been studied enough in the HDI literature.  This could be a future line of research.



\begin{table}[hbtp]
\caption{Ordered clustering  vs HDI development groups: uncertain weights}
\label{results2}
\tiny
\hskip-4.0cm
\begin{tabular}{llll}
\begin{tabular}[t]{c*{\items}{|E}l}
\multicolumn{5}{c}{VERY HIGH DEVELOPMENT}\\\hline 
\multicolumn{1}{c}{Country} & 
\multicolumn{1}{c}{$C_1$} & 
\multicolumn{1}{c}{$C_2$} & 
\multicolumn{1}{c}{$C_3$} & 
\multicolumn{1}{c}{$C_4$} 
\\\hline
1	&	0.00 	&	0.00 	&	0.00 	&	1.00 	 \\\hline
2	&	0.00 	&	0.00 	&	0.00 	&	1.00 	 \\\hline
3	&	0.00 	&	0.00 	&	0.00 	&	1.00 	 \\\hline
4	&	0.00 	&	0.00 	&	0.00 	&	1.00 	 \\\hline
5	&	0.00 	&	0.00 	&	0.00 	&	1.00 	 \\\hline
6	&	0.00 	&	0.00 	&	0.00 	&	1.00 	 \\\hline
7	&	0.00 	&	0.00 	&	0.00 	&	1.00 	 \\\hline
8	&	0.00 	&	0.00 	&	0.00 	&	1.00 	 \\\hline
9	&	0.00 	&	0.00 	&	0.00 	&	1.00 	 \\\hline
10	&	0.00 	&	0.00 	&	0.00 	&	1.00 	 \\\hline
11	&	0.00 	&	0.00 	&	0.00 	&	1.00 	 \\\hline
12	&	0.00 	&	0.00 	&	0.00 	&	1.00 	 \\\hline
13	&	0.00 	&	0.00 	&	0.00 	&	1.00 	 \\\hline
14	&	0.00 	&	0.00 	&	0.00 	&	1.00 	 \\\hline
15	&	0.00 	&	0.00 	&	0.00 	&	1.00 	 \\\hline
16	&	0.00 	&	0.00 	&	0.00 	&	1.00 	 \\\hline
17	&	0.00 	&	0.00 	&	0.00 	&	1.00 	 \\\hline
18	&	0.00 	&	0.00 	&	0.26 	&	0.74 	 \\\hline
19	&	0.00 	&	0.00 	&	0.12 	&	0.88 	 \\\hline
20	&	0.00 	&	0.00 	&	0.00 	&	1.00 	 \\\hline
21	&	0.00 	&	0.00 	&	0.23 	&	0.77 	 \\\hline
22	&	0.00 	&	0.00 	&	0.00 	&	1.00 	 \\\hline
23	&	0.00 	&	0.00 	&	0.00 	&	1.00 	 \\\hline
24	&	0.00 	&	0.00 	&	0.00 	&	1.00 	 \\\hline
25	&	0.00 	&	0.00 	&	0.25 	&	0.75 	 \\\hline
26	&	0.00 	&	0.00 	&	0.00 	&	1.00 	 \\\hline
27	&	0.00 	&	0.00 	&	0.04 	&	0.96 	 \\\hline
28	&	0.00 	&	0.00 	&	0.00 	&	1.00 	 \\\hline
29	&	0.00 	&	0.00 	&	0.15 	&	0.85 	 \\\hline
30	&	0.00 	&	0.00 	&	0.00 	&	1.00 	 \\\hline
31	&	0.00 	&	0.00 	&	0.25 	&	0.75 	 \\\hline
32	&	0.00 	&	0.00 	&	0.05 	&	0.95 	 \\\hline
33	&	0.00 	&	0.00 	&	0.00 	&	1.00 	 \\\hline
34	&	0.00 	&	0.00 	&	0.22 	&	0.78 	 \\\hline
35	&	0.00 	&	0.00 	&	0.84 	&	0.16 	 \\\hline
36	&	0.00 	&	0.00 	&	0.72 	&	0.28 	 \\\hline
37	&	0.00 	&	0.00 	&	0.77 	&	0.23 	 \\\hline
38	&	0.00 	&	0.00 	&	0.62 	&	0.38 	 \\\hline
39	&	0.00 	&	0.00 	&	0.30 	&	0.70 	 \\\hline
40	&	0.00 	&	0.00 	&	0.29 	&	0.71 	 \\\hline
41	&	0.00 	&	0.11 	&	0.75 	&	0.14 	 \\\hline
42	&	0.00 	&	0.00 	&	0.10 	&	0.90 	 \\\hline
43	&	0.00 	&	0.00 	&	1.00 	&	0.00 	 \\\hline
44	&	0.00 	&	0.00 	&	0.66 	&	0.34 	 \\\hline
45	&	0.00 	&	0.00 	&	1.00 	&	0.00 	 \\\hline
46	&	0.00 	&	0.00 	&	0.59 	&	0.41 	 \\\hline
47	&	0.00 	&	0.00 	&	1.00 	&	0.00 	 \\\hline
48	&	0.00 	&	0.00 	&	0.52 	&	0.48 	 \\\hline
49	&	0.00 	&	0.00 	&	0.54 	&	0.46 	 \\\hline
50	&	0.00 	&	0.00 	&	0.61 	&	0.39 	 \\\hline
51	&	0.00 	&	0.00 	&	0.74 	&	0.26 	 \\\hline
52	&	0.00 	&	0.00 	&	0.78 	&	0.22 	 \\\hline
53	&	0.00 	&	0.00 	&	0.84 	&	0.16 	 \\\hline
54	&	0.00 	&	0.00 	&	0.95 	&	0.05 	 \\\hline
55	&	0.00 	&	0.00 	&	0.50 	&	0.50 	 \\\hline
56	&	0.00 	&	0.00 	&	0.79 	&	0.21 	 \\\hline
57	&	0.00 	&	0.27 	&	0.73 	&	0.00 	 \\\hline
58	&	0.00 	&	0.00 	&	0.64 	&	0.36 	 \\\hline
59	&	0.00 	&	0.08 	&	0.71 	&	0.21 	 \\\hline
60	&	0.00 	&	0.00 	&	0.87 	&	0.13 	 \\\hline
61	&	0.00 	&	0.00 	&	1.00 	&	0.00 	 \\\hline
62	&	0.00 	&	0.00 	&	1.00 	&	0.00 	 \\\hline
\end{tabular}

&

\begin{tabular}[t]{c*{\items}{|E}|}
\multicolumn{5}{c}{HIGH DEVELOPMENT}\\\hline 
\multicolumn{1}{c}{Country} & 
\multicolumn{1}{c}{$C_1$} & 
\multicolumn{1}{c}{$C_2$} & 
\multicolumn{1}{c}{$C_3$} & 
\multicolumn{1}{c}{$C_4$} 
\\\hline
63	&	0.00 	&	0.00 	&	0.95 	&	0.05 	 \\\hline
64	&	0.00 	&	0.00 	&	0.96 	&	0.04 	 \\\hline
65	&	0.00 	&	0.00 	&	1.00 	&	0.00 	 \\\hline
66	&	0.00 	&	0.00 	&	1.00 	&	0.00 	 \\\hline
67	&	0.00 	&	0.00 	&	0.98 	&	0.02 	 \\\hline
68	&	0.00 	&	0.00 	&	0.75 	&	0.25 	 \\\hline
69	&	0.00 	&	0.00 	&	0.95 	&	0.05 	 \\\hline
70	&	0.00 	&	0.00 	&	0.57 	&	0.42 	 \\\hline
71	&	0.00 	&	0.00 	&	0.93 	&	0.07 	 \\\hline
72	&	0.00 	&	0.00 	&	0.47 	&	0.53 	 \\\hline
73	&	0.00 	&	0.00 	&	1.00 	&	0.00 	 \\\hline
74	&	0.00 	&	0.05 	&	0.95 	&	0.00 	 \\\hline
75	&	0.00 	&	0.00 	&	1.00 	&	0.00 	 \\\hline
76	&	0.00 	&	0.00 	&	1.00 	&	0.00 	 \\\hline
77	&	0.00 	&	0.06 	&	0.94 	&	0.00 	 \\\hline
78	&	0.00 	&	0.00 	&	0.77 	&	0.23 	 \\\hline
79	&	0.00 	&	0.02 	&	0.98 	&	0.00 	 \\\hline
80	&	0.00 	&	0.00 	&	1.00 	&	0.00 	 \\\hline
81	&	0.00 	&	0.00 	&	0.80 	&	0.20 	 \\\hline
82	&	0.00 	&	0.00 	&	1.00 	&	0.00 	 \\\hline
83	&	0.00 	&	0.00 	&	1.00 	&	0.00 	 \\\hline
84	&	0.00 	&	0.00 	&	1.00 	&	0.00 	 \\\hline
85	&	0.00 	&	0.00 	&	1.00 	&	0.00 	 \\\hline
86	&	0.00 	&	0.00 	&	1.00 	&	0.00 	 \\\hline
87	&	0.00 	&	0.14 	&	0.86 	&	0.00 	 \\\hline
88	&	0.00 	&	0.00 	&	0.90 	&	0.10 	 \\\hline
89	&	0.00 	&	0.00 	&	1.00 	&	0.00 	 \\\hline
90	&	0.00 	&	0.00 	&	1.00 	&	0.00 	 \\\hline
91	&	0.00 	&	0.27 	&	0.73 	&	0.00 	 \\\hline
92	&	0.00 	&	0.20 	&	0.80 	&	0.00 	 \\\hline
93	&	0.00 	&	0.24 	&	0.73 	&	0.03 	 \\\hline
94	&	0.00 	&	0.49 	&	0.51 	&	0.00 	 \\\hline
95	&	0.00 	&	0.00 	&	1.00 	&	0.00 	 \\\hline
96	&	0.00 	&	0.00 	&	1.00 	&	0.00 	 \\\hline
97	&	0.00 	&	0.00 	&	1.00 	&	0.00 	 \\\hline
98	&	0.00 	&	0.16 	&	0.84 	&	0.00 	 \\\hline
99	&	0.00 	&	0.24 	&	0.72 	&	0.04 	 \\\hline
100	&	0.00 	&	0.00 	&	1.00 	&	0.00 	 \\\hline
101	&	0.00 	&	0.00 	&	1.00 	&	0.00 	 \\\hline
102	&	0.00 	&	0.22 	&	0.78 	&	0.00 	 \\\hline
103	&	0.00 	&	0.00 	&	1.00 	&	0.00 	 \\\hline
104	&	0.00 	&	0.70 	&	0.27 	&	0.02 	 \\\hline
105	&	0.00 	&	0.09 	&	0.86 	&	0.05 	 \\\hline
106	&	0.00 	&	0.24 	&	0.76 	&	0.00 	 \\\hline
107	&	0.00 	&	0.53 	&	0.37 	&	0.10 	 \\\hline
108	&	0.00 	&	0.73 	&	0.27 	&	0.00 	 \\\hline
109	&	0.00 	&	0.50 	&	0.38 	&	0.12 	 \\\hline
110	&	0.00 	&	0.66 	&	0.34 	&	0.00 	 \\\hline
111	&	0.00 	&	0.38 	&	0.62 	&	0.00 	 \\\hline
112	&	0.00 	&	0.06 	&	0.94 	&	0.00 	 \\\hline
113	&	0.17 	&	0.36 	&	0.47 	&	0.00 	 \\\hline
114	&	0.00 	&	0.05 	&	0.95 	&	0.00 	 \\\hline
115	&	0.00 	&	0.58 	&	0.42 	&	0.00 	 \\\hline
116	&	0.00 	&	0.76 	&	0.24 	&	0.00 	 \\\hline
\end{tabular}

&

\begin{tabular}[t]{c*{\items}{|E}|}
\multicolumn{5}{c}{MEDIUM DEVELOPMENT}\\\hline 
\multicolumn{1}{c}{Country} & 
\multicolumn{1}{c}{$C_1$} & 
\multicolumn{1}{c}{$C_2$} & 
\multicolumn{1}{c}{$C_3$} & 
\multicolumn{1}{c}{$C_4$} 
\\\hline
117	&	0.00 	&	0.09 	&	0.86 	&	0.04 	 \\\hline
118	&	0.00 	&	0.01 	&	0.99 	&	0.00 	 \\\hline
119	&	0.00 	&	0.00 	&	1.00 	&	0.00 	 \\\hline
120	&	0.00 	&	1.00 	&	0.00 	&	0.00 	 \\\hline
121	&	0.00 	&	0.46 	&	0.54 	&	0.00 	 \\\hline
122	&	0.00 	&	0.07 	&	0.89 	&	0.04 	 \\\hline
123	&	0.00 	&	0.78 	&	0.22 	&	0.00 	 \\\hline
124	&	0.00 	&	0.92 	&	0.08 	&	0.00 	 \\\hline
125	&	0.00 	&	0.64 	&	0.36 	&	0.00 	 \\\hline
126	&	0.00 	&	0.96 	&	0.04 	&	0.00 	 \\\hline
127	&	0.00 	&	0.85 	&	0.15 	&	0.00 	 \\\hline
128	&	0.00 	&	0.77 	&	0.23 	&	0.00 	 \\\hline
129	&	0.00 	&	1.00 	&	0.00 	&	0.00 	 \\\hline
130	&	0.21 	&	0.79 	&	0.00 	&	0.00 	 \\\hline
131	&	0.27 	&	0.73 	&	0.00 	&	0.00 	 \\\hline
132	&	0.01 	&	0.76 	&	0.23 	&	0.00 	 \\\hline
133	&	0.00 	&	1.00 	&	0.00 	&	0.00 	 \\\hline
134	&	0.67 	&	0.33 	&	0.00 	&	0.00 	 \\\hline
135	&	0.00 	&	0.98 	&	0.02 	&	0.00 	 \\\hline
136	&	0.00 	&	1.00 	&	0.00 	&	0.00 	 \\\hline
137	&	0.00 	&	1.00 	&	0.00 	&	0.00 	 \\\hline
138	&	0.13 	&	0.87 	&	0.00 	&	0.00 	 \\\hline
139	&	0.27 	&	0.73 	&	0.00 	&	0.00 	 \\\hline
140	&	0.00 	&	1.00 	&	0.00 	&	0.00 	 \\\hline
141	&	0.00 	&	1.00 	&	0.00 	&	0.00 	 \\\hline
142	&	0.19 	&	0.81 	&	0.00 	&	0.00 	 \\\hline
143	&	0.20 	&	0.80 	&	0.00 	&	0.00 	 \\\hline
144	&	0.82 	&	0.18 	&	0.00 	&	0.00 	 \\\hline
145	&	0.39 	&	0.61 	&	0.00 	&	0.00 	 \\\hline
146	&	0.18 	&	0.82 	&	0.00 	&	0.00 	 \\\hline
147	&	0.00 	&	1.00 	&	0.00 	&	0.00 	 \\\hline
148	&	0.13 	&	0.87 	&	0.00 	&	0.00 	 \\\hline
149	&	0.52 	&	0.48 	&	0.00 	&	0.00 	 \\\hline
150	&	0.24 	&	0.75 	&	0.01 	&	0.00 	 \\\hline
151	&	0.44 	&	0.45 	&	0.11 	&	0.00 	 \\\hline
152	&	0.53 	&	0.47 	&	0.00 	&	0.00 	 \\\hline
153	&	0.03 	&	0.94 	&	0.04 	&	0.00 	 \\\hline
\end{tabular}

&

\begin{tabular}[t]{c*{\items}{|E}|}
\multicolumn{5}{c}{LOW DEVELOPMENT}\\\hline 
\multicolumn{1}{c}{Country} & 
\multicolumn{1}{c}{$C_1$} & 
\multicolumn{1}{c}{$C_2$} & 
\multicolumn{1}{c}{$C_3$} & 
\multicolumn{1}{c}{$C_4$} 
\\\hline
154	&	0.56 	&	0.44 	&	0.00 	&	0.00 	 \\\hline
155	&	1.00 	&	0.00 	&	0.00 	&	0.00 	 \\\hline
156	&	0.73 	&	0.27 	&	0.00 	&	0.00 	 \\\hline
157	&	0.34 	&	0.66 	&	0.00 	&	0.00 	 \\\hline
158	&	0.79 	&	0.21 	&	0.00 	&	0.00 	 \\\hline
159	&	0.55 	&	0.45 	&	0.00 	&	0.00 	 \\\hline
160	&	0.29 	&	0.71 	&	0.00 	&	0.00 	 \\\hline
161	&	1.00 	&	0.00 	&	0.00 	&	0.00 	 \\\hline
162	&	0.00 	&	1.00 	&	0.00 	&	0.00 	 \\\hline
163	&	0.77 	&	0.23 	&	0.00 	&	0.00 	 \\\hline
164	&	0.64 	&	0.36 	&	0.00 	&	0.00 	 \\\hline
165	&	1.00 	&	0.00 	&	0.00 	&	0.00 	 \\\hline
166	&	0.85 	&	0.15 	&	0.00 	&	0.00 	 \\\hline
167	&	0.64 	&	0.36 	&	0.00 	&	0.00 	 \\\hline
168	&	1.00 	&	0.00 	&	0.00 	&	0.00 	 \\\hline
169	&	0.85 	&	0.15 	&	0.00 	&	0.00 	 \\\hline
170	&	1.00 	&	0.00 	&	0.00 	&	0.00 	 \\\hline
171	&	0.89 	&	0.11 	&	0.00 	&	0.00 	 \\\hline
172	&	0.92 	&	0.08 	&	0.00 	&	0.00 	 \\\hline
173	&	0.90 	&	0.10 	&	0.00 	&	0.00 	 \\\hline
174	&	1.00 	&	0.00 	&	0.00 	&	0.00 	 \\\hline
175	&	1.00 	&	0.00 	&	0.00 	&	0.00 	 \\\hline
176	&	1.00 	&	0.00 	&	0.00 	&	0.00 	 \\\hline
177	&	0.91 	&	0.09 	&	0.00 	&	0.00 	 \\\hline
178	&	0.97 	&	0.03 	&	0.00 	&	0.00 	 \\\hline
179	&	0.67 	&	0.33 	&	0.00 	&	0.00 	 \\\hline
180	&	1.00 	&	0.00 	&	0.00 	&	0.00 	 \\\hline
181	&	1.00 	&	0.00 	&	0.00 	&	0.00 	 \\\hline
182	&	1.00 	&	0.00 	&	0.00 	&	0.00 	 \\\hline
183	&	0.93 	&	0.07 	&	0.00 	&	0.00 	 \\\hline
184	&	1.00 	&	0.00 	&	0.00 	&	0.00 	 \\\hline
185	&	0.89 	&	0.11 	&	0.00 	&	0.00 	 \\\hline
186	&	1.00 	&	0.00 	&	0.00 	&	0.00 	 \\\hline
187	&	1.00 	&	0.00 	&	0.00 	&	0.00 	 \\\hline
188	&	1.00 	&	0.00 	&	0.00 	&	0.00 	 \\\hline
189	&	1.00 	&	0.00 	&	0.00 	&	0.00 	 \\\hline
\end{tabular}

\end{tabular}

\end{table}





\section{Conclusions}\label{conclusions}
  
In this paper, a non-compensatory and stochastic multi-criteria clustering approach is applied to group countries included in the Human Development Index Report. Results obtained with this approach are compared to the deterministic 2018 HDI ranking. Findings show that uncertainty shapes results, but it depends on the variable in which randomness is considered. Our approach is based on the outranking relation, which uses two preferential parameters: the indifference and the preference thresholds, and the weights of criteria.  We have found that uncertainty in thresholds leads to more ambiguity in outcomes than uncertainty in weights.

A limitation of our approach is the usage of $K$-means like algorithm because results are dependent on the initial set of cluster centers.  To mitigate this drawback, we have used some heuristics to find the number  and initial centroids. Anyway, future research is needed to  consider this feature in contexts of stochastic variables.
     
%\section*{Acknowledgement}

          
\section*{References}
\bibliographystyle{authoryear}
\bibliography{../bib/mybib-mcdm,../bib/mybib-mcdm-v1,../bib/mybib}





%\appendix
%\section{Separability condition}\label{sepcondition}


\end{document}

